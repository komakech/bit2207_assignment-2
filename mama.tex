\documentclass[11pt]{article}
\title{AUTOMATIC VERIFICATION OF COMPLEX SYSTEMS IN A ENERGY SECTOR}
\author{KOMAKECH GEOFFREY,MAGERO TIMOTHY, 	AINEBYOONA ALVINE,TAYEBWA HAROLD}
\begin{document}
\maketitle
\section{Abstract}
Power and energy networks are systems of great societal and economic relevance and impact, particularly given the recent growing emphasis on environmental issues and on sustainable sub- stitutes (renewable) to traditional energy sources (coal, oil, nuclear). Power networks also stitutes (renewable) to traditional energy sources (coal, oil, nuclear). Power networks also represent systems of considerable engineering interest. 
\section{Introduction}
\subsection{Background}
\cite{ru}

Power and energy networks are systems of great societal and economic relevance and impact, particularly given the recent growing emphasis on environmental issues and on sustainable sub- stitutes (renewable) to traditional energy sources (coal, oil, nuclear). Power networks also stitutes (renewable) to traditional energy sources (coal, oil, nuclear). Power networks also represent systems of considerable engineering interest.
\space
\newline

These reprecomplex manner they are heterogeneous, that is they can be naturally modelled through a sent systems of considerable engineering interest, since: they can be large-scale and can involve numbers of various devices interconnected in a combination of continuous dynamical elements (to capture the evolution of quantities such as voltages,frequencies and generation output) and discrete dynamical components (to capture changes in the network topology,controller logic, state of breakers, isolation devices,transformer taps, etc.)

\section{Research Objective}
The aim of this the aims of this concept paper has been to survey existing and explore novel formal Frameworks for modeling, analysis and control of complex, large scale cyber-physical systems, with emphasis on applications in power networks.
\section{Specific Objective}
\begin{enumerate}
\item 	To provide continuous dynamics models the evolution of voltages, frequencies, etc.
\item 	To provide discrete dynamics models controller logic and changes in network topology (unit commitment
\item 	To provide probability models the uncertainty about power demand, power supply from renewable and power market price.
\item  To provide solar installation contribute to the energy sector
\end{enumerate}

\section{Problems Statement}
Though there are many ways which has been put in place to solved the problems facing the energy sector and to cope up with  the demand and supply of electricity such as constructing power dams like bujagali, karuma  on the process especially in Uganda but , Currently we are facing the transition of the traditional Energy world to a Smart Energy System. Flexibility is key in Smart Energy. Flexibility here is not about the flow of energy itself but about the willingness or ability of energy units (i. e. households, generators, battery elements) to vary respectively in their consumption, production or storage. The important of solving this problems is stabilize the power supply to match with the demand for electricity.  
\section{Research Scope}
The research scope limited to the people in rural area in Uganda since they  lack accessed to electricity and  there is needs to address the problem of power shortage.
\section{Research Significant}
\begin{enumerate}
\item This study is important since it helps the people in rural area to accessed electricity in relatively low cost or no cost at all.
\item  This study is also important because it improves on environmental conservation by reducing on deforestation since almost all people in rural area uses firewood as their source of energy and provision of alternative would reduce on that.
\end{enumerate}
\section{Methodology}
\subsection{Method and Technigue}
The proposed task, we shall advise the government to er to rural people since sunshine is provided natural by God.
\cite{ru}
\cite{di}
\cite{kn}
install solar to provide pow
\bibliography{waca}
\bibliographystyle{plain}
\end{document}